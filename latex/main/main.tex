\documentclass[11pt,a4paper]{amsart}
\usepackage[numbers]{natbib}
\usepackage{stix}
\usepackage{siunitx}
\usepackage{booktabs}
\usepackage{overpic}
\usepackage{url}
\urlstyle{sf}
\usepackage{xspace}
\usepackage{xcolor}
\usepackage{listings}
\lstset{basicstyle=\ttfamily,
  showstringspaces=false,
  commentstyle=\color{red},
  keywordstyle=\color{blue}
}
\colorlet{punct}{red!60!black}
\definecolor{background}{HTML}{EEEEEE}
\definecolor{delim}{RGB}{20,105,176}
\colorlet{numb}{magenta!60!black}
\lstdefinelanguage{json}{
    basicstyle=\normalfont\ttfamily,
    numbers=left,
    numberstyle=\scriptsize,
    stepnumber=1,
    numbersep=8pt,
    showstringspaces=false,
    breaklines=true,
    frame=lines,
    backgroundcolor=\color{background},
    literate=
     *{0}{{{\color{numb}0}}}{1}
      {1}{{{\color{numb}1}}}{1}
      {2}{{{\color{numb}2}}}{1}
      {3}{{{\color{numb}3}}}{1}
      {4}{{{\color{numb}4}}}{1}
      {5}{{{\color{numb}5}}}{1}
      {6}{{{\color{numb}6}}}{1}
      {7}{{{\color{numb}7}}}{1}
      {8}{{{\color{numb}8}}}{1}
      {9}{{{\color{numb}9}}}{1}
      {:}{{{\color{punct}{:}}}}{1}
      {,}{{{\color{punct}{,}}}}{1}
      {\{}{{{\color{delim}{\{}}}}{1}
      {\}}{{{\color{delim}{\}}}}}{1}
      {[}{{{\color{delim}{[}}}}{1}
      {]}{{{\color{delim}{]}}}}{1},
}
\usepackage{bm}
\usepackage[implicit=false,colorlinks=true,urlcolor=blue]{hyperref}
\usepackage{soul}
\usepackage{xparse}
\usepackage{xcolor}
\definecolor{Light}{gray}{.90}
\sethlcolor{Light}
\usepackage{amsmath}

\newcommand{\cholMatrix}{\ensuremath{\boldsymbol{L}}}
\newcommand{\cholMatrixT}{\ensuremath{\boldsymbol{L}^{\text{T}}}}
\newcommand{\covMatrix}{\ensuremath{\boldsymbol{C}}}
\newcommand{\covMatrixElem}{\ensuremath{c}}
\newcommand{\cotwo}{\ensuremath{\text{CO}_2}}
\DeclareMathOperator{\covarianceFuncXY}{Cov}
\DeclareMathOperator{\covarianceFuncH}{C}
\newcommand{\deltaH}{\ensuremath{\Delta h}}
\newcommand{\displacement}{\ensuremath{\boldsymbol{h}}}
\newcommand{\distance}{\ensuremath{h}}
\newcommand{\distMatrix}{\ensuremath{\boldsymbol{H}}}
\newcommand{\docMain}{\texttt{main.pdf}}
\newcommand{\docSetup}{\hltexttt{setup.pdf}}
\newcommand{\docVariogram}{\hltexttt{variogram.pdf}}
\newcommand{\docCholesky}{\hltexttt{cholesky.pdf}}
\newcommand{\docsFolder}{\hltexttt{docs}}
\newcommand{\domainCylCenter}{\ensuremath{P}}
\newcommand{\domainCurvature}{\ensuremath{\gamma}}
\newcommand{\domainCylRadius}{\ensuremath{r}}
\newcommand{\domainWidth}{\ensuremath{w}}
\newcommand{\domainLength}{\ensuremath{l}}
\newcommand{\domainHeight}{\ensuremath{h}}
\newcommand{\domainAngleHeight}{\ensuremath{H}}
\newcommand{\domainXA}{\ensuremath{\boldsymbol{x}_1}}
\newcommand{\domainXB}{\ensuremath{\boldsymbol{x}_2}}
\newcommand{\domainPoint}{\ensuremath{\boldsymbol{x}}}
\DeclareMathOperator{\expectedValue}{E}
\newcommand{\heightBottom}{\ensuremath{\mathcal{H}_{\text{ref}}}}
\newcommand{\heightRandomReal}{\ensuremath{\hat{h}}}
\newcommand{\heightRandom}{\ensuremath{\tilde{H}}}
\newcommand{\heightTop}{\ensuremath{\mathcal{H}_{\text{top}}}}
\newcommand{\heightTopBase}{\ensuremath{\mathcal{H}_{\text{top,0}}}}
\newcommand{\hltexttt}[1]{\texttt{\hl{#1}}}
\newcommand{\hmin}{\ensuremath{h_{\text{min}}}}
\newcommand{\hmax}{\ensuremath{h_{\text{max}}}}
\newcommand{\identityMatrix}{\ensuremath{\boldsymbol{I}}}
\newcommand{\jsonInputParamFile}{\hltexttt{input\_params.json}}
\newcommand{\jsonParam}[1]{\ensuremath{\text{\hltexttt{#1}}}}
\newcommand{\mainScript}{\hltexttt{gen\_realizations\_and\_run\_vesa.py}}
\newcommand{\mainScriptKJ}{\hltexttt{gen\_vesa\_input\_file.py}}
\newcommand{\mean}{\ensuremath{\mu}}
\newcommand{\numDomainPoints}{\ensuremath{N}}
\newcommand{\paramAmp}{\scriptParam{requested\_delta\_h}}
\newcommand{\paramNumReal}{\scriptParam{num\_real}}
\newcommand{\paramStd}{\scriptParam{std}}
\newcommand{\pythonModule}[1]{\hltexttt{#1}}
\newcommand{\pythonPip}{\hltexttt{pip}}
\newcommand{\pythScriptName}{\hltexttt{gen\_vesa\_input\_files.py}}
\newcommand{\radDist}{\ensuremath{r}}
\newcommand{\randVarZ}{\ensuremath{\tilde{Z}}}
\newcommand{\randVecZ}{\ensuremath{\tilde{\boldsymbol{Z}}}}
\newcommand{\randVecz}{\ensuremath{\tilde{\boldsymbol{z}}}}
\newcommand{\range}{\ensuremath{\kappa}}
\newcommand{\rangeX}{\scriptParam{range\_x}}
\newcommand{\rangeY}{\scriptParam{range\_y}}
\newcommand{\reservoirThickness}{\ensuremath{\mathcal{W}}}
\newcommand{\scriptParam}[1]{\ensuremath{\hltexttt{#1}}}
\newcommand{\stdev}{\ensuremath{\sigma}}
\newcommand{\tiltAngle}{\ensuremath{\alpha}}
\newcommand{\unit}[1]{\ensuremath{[\si{#1}]}}
\newcommand{\variance}{\ensuremath{\sigma^2}}
\newcommand{\varianceFunc}{\ensuremath{\text{Var}}}
\newcommand{\variogram}{\ensuremath{\gamma}}
\newcommand{\variogramVariance}{\ensuremath{\sigma_v^2}}
\newcommand{\vesaVariable}[1]{\ensuremath{\text{\hltexttt{#1}}}}
\newcommand{\vesaGitHub}{\url{https://github.com/NORCE-Energy/vesa}}
\newcommand{\vesaOutFile}[1]{\hltexttt{#1\_out}}

\DeclareDocumentCommand \vesaInFile { m o } {
  \IfNoValueTF {#2}%
    {\hltexttt{#1.IN}}%
    {\hltexttt{#1.IN}[#2]}%
}


\newenvironment{ovpc}[2]{\begin{overpic}[width=#1\textwidth]{../fig/#2}}{\end{overpic}}
\begin{document}
\title{Python script to generate top-surfaces to be used with the VESA simulator}
\date{\today}
\maketitle
\section{Introduction}
\label{sec:introduction}
In reservoir simulation, the traditional approach to study the impact of
geology on fluid migration is 
to create multiple realizations of a geostatistical model and then run a
forward simulator on each realization. An alternative approach is to use a
machine learning method. Here, a ``prediction tool'' is generated after an initial
training session on sample data. After the training, the tool can be
used to predict the outcome of, e.g., \cotwo{} migration without having to run
a traditional reservoir simulator (which should improve the efficiency of prediction).

In this study, we consider how the top surface morphology of the
reservoir can impact storage capacity \cite{nil12:imp} of \cotwo{}.
A statistical model for generating top surfaces of a
reservoir is developed. Sampling from this random field provides 
realizations that can be used as input for the VESA simulator
\cite{gas09:ver} to produce breakthrough curves. In this way, we obtain a
mapping between the parameters used in the statistical model of the top
surface and the corresponding
breakthrough curves, providing training data the machine learning algorithm.

The main focus of this documentation is to describe how to generate the top
surfaces. It also briefly mention how to run the VESA simulator.
It does not describe the actual machine learning procedure. 

\section{Documentation}
To create the PDF documentation files from the \LaTeX{} source files,
run \shellCommand{make docs} from the top folder. This will install the documentation files in the \docsFolder{}.
The current document, \docMain{}, gives an overview of the main Python script,
further documentation is available in the \docsFolder{} folder:  
\begin{itemize}
\item \docSetup {} : describes the top surface model and its parameters in
  more detail, 
\item \docVariogram{} : describes the model for the covariance function,
\item \docCholesky{} : a short note on how the Cholesky factorization is used
  to help generate samples from the random field.
\end{itemize}
\section{Prerequisites}
\subsection{Python installation}
The script uses Python version 3.
%The minimum version supported is version
%3.3, due to the usage of module \pythonModule{shutil}.
The following non-standard Python
packages (which should be straightforward to install with the \pythonPip{} 
\cite{pyt18:pip} command) are used:
\begin{itemize}
\item \hltexttt{jsoncomment}
\item \hltexttt{numpy}
\end{itemize}
\subsection{VESA simulator}
The VESA simulator can be downloaded from GitHub at

\[ \text{\vesaGitHub{}} \]

To install it, you can try
\begin{lstlisting}[language=bash]
$ git clone https://github.com/NORCE-Energy/vesa.git
$ cd vesa
$ ./configure
$ make
\end{lstlisting}


\section{Scripts and modules}
The Python scripts and modules are located in the folder \hltexttt{python}:
\begin{itemize}
\item \mainScriptKJ{} : This is the main script to be used as a component in the
  machine learning algorithm.
\item The part of the code that generates a single realization has been
  separated out into a Python module \pythonModule{norce.bigq5}. The file of
  this module is located in folder \hltexttt{python/norce/bigq5.py}.
\item \jsonInputParamFile{} : Some parameters to the main script
  \mainScriptKJ{} are given at the command line, and the rest of the
  parameters are given in the JSON file \jsonInputParamFile{}, see below.
\end{itemize}

\section{Main script}
The general command line syntax for calling the main Python script\\
\mainScriptKJ{} is:
\begin{lstlisting}[language=bash,basicstyle=\scriptsize]
$ gen_vesa_input_file.py <range_x> <range_y> <requested_delta_h>
\end{lstlisting}
where the leading \hltexttt{\$} indicates a Shell prompt, and
\begin{itemize}
\item \rangeX{}, is the $x$-range of the variogram,
\item \rangeY{}, is the $y$-range of the variogram,
\item \paramAmp{}, is the requested height variation (amplitude) of the ridges,
\end{itemize}
See the documentation file \docSetup{} for more information about the
background for the parameters.

For example the following command line:
\begin{lstlisting}[language=bash]
$ gen_vesa_input_file.py 5000 40000 50
\end{lstlisting}
will generate a realization from the stochastic field. The covariance of
the field will have a range of 5000 in the $x$-direction and a range of 40000 in
the $y$-direction. The amplitude of the covariance function is set such that
we have high confidence (the level of confidence is currently set to 99.7\%,
but this can be changed in the parameter file, see discussion 
below) that a sample point drawn from the field will lie in the interval $[-50, 50]$.

\subsection{Fixed parameters}
The three parameters, \rangeX{}, \rangeY{}, and\\ \paramAmp{}, described in the
previous section, can be called ``dynamic'' 
since they can be changed conveniently on the command line, the rest of the
parameters to the script are given in a JSON file \jsonInputParamFile{} that
should be located in the same folder as the main script. The content of this
file is:
\begin{lstlisting}[language=json,firstnumber=1]
"domain" : {
   "curvature" : 0.005,
   "tilt_angle" : 0.477
},
"num_std_dev_confidence_delta_h" : 3, 
"nx" : 200,
"ny" : 125,
"output_dir" : "../vesa_case_dir",
"reservoir_depth" : 1000,
"reservoir_thickness" : 100,
"reservoir_xy_size" : [50000, 25000],
"variogram" : {
   "type" : "spherical",
},
\end{lstlisting}
Here,
\begin{itemize}

\item \jsonParam{domain['curvature']} is the curvature of the domain, see
  documentation file \docSetup{} for more information. 

\item \jsonParam{domain['tilt\_angle']} is the domain tilt angle, see
  \docSetup{} for more information. 

\item The parameter \jsonParam{num\_std\_dev\_confidence\_delta\_h} describes
  the confidence level of the command line parameter \paramAmp{}.
  Let $N_{\stdev} := \jsonParam{num\_std\_dev\_confidence\_delta\_h}$, then
  the generated random field is stationary with a standard
  deviation given by
  \begin{align*}
    \sigma = \frac{\paramAmp}{N_{\stdev}}.
  \end{align*}
  For example if $N_{\stdev} = 3$, then according to the empirical
rule \cite{wik18:ers}, there will be a 99.7\% probability that a sampled value
will lie in the interval $[-\paramAmp, \paramAmp]$.

\item  The parameter \jsonParam{reservoir\_xy\_size} is an array of two
    elements where the first is the $x$-size (the length) of the reservoir, and the
    second specifies the $y$-size (the width) of the reservoir.

\item The parameters \jsonParam{nx} and  \jsonParam{ny} specifies the number
    of grid cells in the $x$- and $y$-directions, respectively. The size of a
    grid cell $(\Delta x, \Delta y)$ is then
    \begin{align*}
      \label{eq:4}
      \Delta x = (\jsonParam{x\_max} - \jsonParam{x\_min})/ \jsonParam{nx}\\
      \Delta y = (\jsonParam{y\_max} - \jsonParam{y\_min})/ \jsonParam{ny}
    \end{align*}
    and the generated surface heights corresponds to the grid cell centers
    (not the grid vertices). 

\item \jsonParam{output\_dir} contains the directory name where the output
  file \vesaInFile{REFHT} (a VESA input file) will be written. The default
  value (if the parameter 
  is not given) is 
  the current directory, ``\hltexttt{.}'' (i.e. a string consisting of a the single
  dot character). 

\item \jsonParam{reservoir\_depth} specifies the depth (below sea level) of
  the lowest point of 
  the bottom surface. This number should be a positive number, although its
  $z$-coordinate will be its negative value.

\item \jsonParam{reservoir\_thickness} gives the thickness of the reservoir.
  Note that this value should agree with the value in the VESA input file \vesaInFile{H}.

\item \jsonParam{variogram['type']} the type of variogram to use when
  generating the covariance matrix, see the documentation file
  \docVariogram{} for more information.

\end{itemize}

\bibliographystyle{plain}
\begin{thebibliography}{1}

\bibitem{wik18:ers}
68–95–99.7 rule.
\newblock
  {\url{https://en.wikipedia.org/wiki/68%E2%80%9395%E2%80%9399.7_rule}}.

\bibitem{pyt18:pip}
Installing packages.
\newblock {\url{https://packaging.python.org/tutorials/installing-packages}}.

\bibitem{gas09:ver}
S.E. Gasda, J.M. Nordbotten, and M.A. Celia.
\newblock Vertical equilibrium with sub-scale analytical methods for geological
  {CO2} sequestration.
\newblock {\em Comput. Geosci.}, 13(4):469--481, 2009.

\bibitem{nil12:imp}
H.M. Nilsen, A.R. Syversveen, K.-A. Lie, J.~Tveranger, and J.M. Nordbotten.
\newblock Impact of top-surface morphology on {CO2} storage capacity.
\newblock {\em International Journal of Greenhouse Gas Control}, 11:221--235,
  2012.

\end{thebibliography}

\end{document}
