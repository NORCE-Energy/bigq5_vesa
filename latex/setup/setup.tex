\documentclass[11pt,a4paper]{amsart}
\usepackage[numbers]{natbib}
\usepackage{stix}
\usepackage{siunitx}
\usepackage{booktabs}
\usepackage{overpic}
\usepackage{xspace}
\usepackage{bm}
\usepackage[implicit=false,colorlinks=true,urlcolor=blue]{hyperref}
\usepackage{soul}
\usepackage{xparse}
\usepackage{xcolor}
\definecolor{Light}{gray}{.90}
\sethlcolor{Light}
\usepackage{amsmath}

\newcommand{\cholMatrix}{\ensuremath{\boldsymbol{L}}}
\newcommand{\cholMatrixT}{\ensuremath{\boldsymbol{L}^{\text{T}}}}
\newcommand{\covMatrix}{\ensuremath{\boldsymbol{C}}}
\newcommand{\covMatrixElem}{\ensuremath{c}}
\newcommand{\cotwo}{\ensuremath{\text{CO}_2}}
\DeclareMathOperator{\covarianceFuncXY}{Cov}
\DeclareMathOperator{\covarianceFuncH}{C}
\newcommand{\deltaH}{\ensuremath{\Delta h}}
\newcommand{\displacement}{\ensuremath{\boldsymbol{h}}}
\newcommand{\distance}{\ensuremath{h}}
\newcommand{\distMatrix}{\ensuremath{\boldsymbol{H}}}
\newcommand{\docMain}{\texttt{main.pdf}}
\newcommand{\docSetup}{\hltexttt{setup.pdf}}
\newcommand{\docVariogram}{\hltexttt{variogram.pdf}}
\newcommand{\docCholesky}{\hltexttt{cholesky.pdf}}
\newcommand{\docsFolder}{\hltexttt{docs}}
\newcommand{\domainCylCenter}{\ensuremath{P}}
\newcommand{\domainCurvature}{\ensuremath{\gamma}}
\newcommand{\domainCylRadius}{\ensuremath{r}}
\newcommand{\domainWidth}{\ensuremath{w}}
\newcommand{\domainLength}{\ensuremath{l}}
\newcommand{\domainHeight}{\ensuremath{h}}
\newcommand{\domainAngleHeight}{\ensuremath{H}}
\newcommand{\domainXA}{\ensuremath{\boldsymbol{x}_1}}
\newcommand{\domainXB}{\ensuremath{\boldsymbol{x}_2}}
\newcommand{\domainPoint}{\ensuremath{\boldsymbol{x}}}
\DeclareMathOperator{\expectedValue}{E}
\newcommand{\heightBottom}{\ensuremath{\mathcal{H}_{\text{ref}}}}
\newcommand{\heightRandomReal}{\ensuremath{\hat{h}}}
\newcommand{\heightRandom}{\ensuremath{\tilde{H}}}
\newcommand{\heightTop}{\ensuremath{\mathcal{H}_{\text{top}}}}
\newcommand{\heightTopBase}{\ensuremath{\mathcal{H}_{\text{top,0}}}}
\newcommand{\hltexttt}[1]{\texttt{\hl{#1}}}
\newcommand{\hmin}{\ensuremath{h_{\text{min}}}}
\newcommand{\hmax}{\ensuremath{h_{\text{max}}}}
\newcommand{\identityMatrix}{\ensuremath{\boldsymbol{I}}}
\newcommand{\jsonInputParamFile}{\hltexttt{input\_params.json}}
\newcommand{\jsonParam}[1]{\ensuremath{\text{\hltexttt{#1}}}}
\newcommand{\mainScript}{\hltexttt{gen\_realizations\_and\_run\_vesa.py}}
\newcommand{\mainScriptKJ}{\hltexttt{gen\_vesa\_input\_file.py}}
\newcommand{\mean}{\ensuremath{\mu}}
\newcommand{\numDomainPoints}{\ensuremath{N}}
\newcommand{\paramAmp}{\scriptParam{requested\_delta\_h}}
\newcommand{\paramNumReal}{\scriptParam{num\_real}}
\newcommand{\paramStd}{\scriptParam{std}}
\newcommand{\pythonModule}[1]{\hltexttt{#1}}
\newcommand{\pythonPip}{\hltexttt{pip}}
\newcommand{\pythScriptName}{\hltexttt{gen\_vesa\_input\_files.py}}
\newcommand{\radDist}{\ensuremath{r}}
\newcommand{\randVarZ}{\ensuremath{\tilde{Z}}}
\newcommand{\randVecZ}{\ensuremath{\tilde{\boldsymbol{Z}}}}
\newcommand{\randVecz}{\ensuremath{\tilde{\boldsymbol{z}}}}
\newcommand{\range}{\ensuremath{\kappa}}
\newcommand{\rangeX}{\scriptParam{range\_x}}
\newcommand{\rangeY}{\scriptParam{range\_y}}
\newcommand{\reservoirThickness}{\ensuremath{\mathcal{W}}}
\newcommand{\scriptParam}[1]{\ensuremath{\hltexttt{#1}}}
\newcommand{\stdev}{\ensuremath{\sigma}}
\newcommand{\tiltAngle}{\ensuremath{\alpha}}
\newcommand{\unit}[1]{\ensuremath{[\si{#1}]}}
\newcommand{\variance}{\ensuremath{\sigma^2}}
\newcommand{\varianceFunc}{\ensuremath{\text{Var}}}
\newcommand{\variogram}{\ensuremath{\gamma}}
\newcommand{\variogramVariance}{\ensuremath{\sigma_v^2}}
\newcommand{\vesaVariable}[1]{\ensuremath{\text{\hltexttt{#1}}}}
\newcommand{\vesaGitHub}{\url{https://github.com/NORCE-Energy/vesa}}
\newcommand{\vesaOutFile}[1]{\hltexttt{#1\_out}}

\DeclareDocumentCommand \vesaInFile { m o } {
  \IfNoValueTF {#2}%
    {\hltexttt{#1.IN}}%
    {\hltexttt{#1.IN}[#2]}%
}


\newenvironment{ovpc}[2]{\begin{overpic}[width=#1\textwidth]{../fig/#2}}{\end{overpic}}
\usepackage{listings}
\usepackage{xcolor}
\colorlet{punct}{red!60!black}
\definecolor{background}{HTML}{EEEEEE}
\definecolor{delim}{RGB}{20,105,176}
\colorlet{numb}{magenta!60!black}
\lstdefinelanguage{json}{
    basicstyle=\normalfont\ttfamily,
    numbers=left,
    numberstyle=\scriptsize,
    stepnumber=1,
    numbersep=8pt,
    showstringspaces=false,
    breaklines=true,
    frame=lines,
    backgroundcolor=\color{background},
    literate=
     *{0}{{{\color{numb}0}}}{1}
      {1}{{{\color{numb}1}}}{1}
      {2}{{{\color{numb}2}}}{1}
      {3}{{{\color{numb}3}}}{1}
      {4}{{{\color{numb}4}}}{1}
      {5}{{{\color{numb}5}}}{1}
      {6}{{{\color{numb}6}}}{1}
      {7}{{{\color{numb}7}}}{1}
      {8}{{{\color{numb}8}}}{1}
      {9}{{{\color{numb}9}}}{1}
      {:}{{{\color{punct}{:}}}}{1}
      {,}{{{\color{punct}{,}}}}{1}
      {\{}{{{\color{delim}{\{}}}}{1}
      {\}}{{{\color{delim}{\}}}}}{1}
      {[}{{{\color{delim}{[}}}}{1}
      {]}{{{\color{delim}{]}}}}{1},
}
\begin{document}
\title{Generation of top-surfaces for the VESA simulator}
\date{\today}
\maketitle
\section{Introduction}
\label{sec:introduction}
A statistical model for generating top surfaces of a
reservoir is described. The surfaces are created as realizations of a random field with
a multivariate normal distribution with given mean, variance, and  
variogram. The model can be applied as a component in a reservoir modeling
procedure. The full reservoir model can then be applied in numerical
simulation of \cotwo{}  
storage to study e.g. how the top surface morphology of the
reservoir can impact storage 
capacity \cite{nil12:imp}. 

First, a rectangular base-case of length $\domainLength$ and width
$\domainWidth$ is constructed, see Figure \ref{fig:dom}.  
\begin{figure}
  \centering
  \begin{ovpc}{.8}{domain/domain2}
  \end{ovpc}
  \caption{A rectangular domain with a curved top surface. The domain is tilted an
  angle \tiltAngle{} from the horizontal. Her \domainLength{} is the length of
  the domain,  \domainWidth{} is the width of the domain, and \domainHeight{} is
  the height. }
  \label{fig:dom}
\end{figure}
Typical values for the domain parameters are given in Table \ref{tab:1}. The
reservoir is tilted in the west-east-direction with a corresponding elevation
height $\domainAngleHeight$. Given this height and the length of the
reservoir (see Table \ref{tab:1}), we can 
compute the tilt angle \tiltAngle{} of the domain as
\begin{align}
  \label{eq:1}
  \tiltAngle = \frac{180}{\pi}\sin^{-1}\frac{\domainAngleHeight}{\domainLength}
  \approx 0.477\; ^\circ
\end{align}
\begin{table}
  \setlength{\tabcolsep}{12pt}
  \centering
  \begin{tabular}{ccc}
    \toprule
    variable & value & unit  \\
    \midrule
    \domainAngleHeight{} & 0.5 & \unit{km} \\
    \domainLength{} & 60 & \unit{km} \\
    \domainWidth{} & 30 & \unit{km} \\
    \domainHeight{} & 0.1 & \unit{km} \\
    \bottomrule
  \end{tabular}
  \caption{Domain parameters}
  \label{tab:1}
\end{table}
When simulating \cotwo{}
storage, the injected \cotwo{} will be less dense than the resident
fluid. Hence, it will move upwards against
the top of the formation. When trapped by the top surface morphology it
will spread horizontally along the short axis until the trap is saturated
with \cotwo{}. To prevent too much horizontal spreading (i.e \cotwo{} reaching
the domain boundary), the top surface is made slightly curved. 

The curvature
is created by fitting the surface to the shape of a cylinder with a diameter
that is larger than the lateral dimension of the reservoir. See Figure \ref{fig:fcyl}.
\begin{figure}
  \centering
  \begin{ovpc}{.5}{domain/fitcyl}
  \end{ovpc}
  \caption{Cross section of a cylinder in the $xz$-plane. We compute the
    coordinate of the center of the cylinder \domainCylCenter{}, given its radius
    \domainCylRadius{} and the
    reservoir top surface 
    lateral dimension (given by the points $\domainXA$ and $\domainXB$). Note
    that the points \domainXA{} and \domainXB{} are constrained to the
    boundary of the cylinder. }
  \label{fig:fcyl}
\end{figure}
We define the curvature $0 \leq \domainCurvature \leq 1$ of the top surface as
\begin{align}
  \label{eq:2}
  \domainCurvature = \frac{\|\domainXB - \domainXA \|}{2\domainCylRadius},
  \qquad 2\domainCylRadius \ge \|\domainXB - \domainXA \|.
\end{align}
The largest curvature occur when the diameter of the cylinder coincides with
the lateral dimension of the reservoir, i.e.
$2\domainCylRadius = \|\domainXB - \domainXA \|$.

\subsection{Adding stochastic features to the top surface}
\label{sec:adding-stoch-feat}
After having created a top surfaces with a given tilt and curvature as
described in the previous section, we add some more (small-scale) stochastic
features to emulate
% morphological features like
offshore sand ridges \cite{nil12:imp}.

As a starting point for the stochastic model,
we assume that the depth (i.e. the negative $z$-coordinate assuming
$z=0$ corresponds to sea level) of the top surface of the reservoir is
given as
\begin{align}
  \label{eq:17}
  \heightTop(\boldsymbol{x}) = \heightTopBase(\boldsymbol{x})
  + \heightRandomReal(\boldsymbol{x})
\end{align}
here $\boldsymbol{x} = (x, y)$ is a point in the $xy$-plane,
$\heightTopBase(\boldsymbol{x})$ is the top surface as described in Section
\ref{sec:introduction}, and
where $\heightRandomReal(\boldsymbol{x})$ is a realization of a random field
$\heightRandom(\boldsymbol{x})$ with zero mean and a
standard deviation that is much smaller than the reservoir thickness.
%$\reservoirThickness(\boldsymbol{x})$.
%Also, note the bottom surface of the reservoir is now given by
%\begin{align}
%  \label{eq:8}
%  \heightBottom(\boldsymbol{x}) = \heightTop(\boldsymbol{x})
%  - \reservoirThickness(\boldsymbol{x})
%\end{align}
Hence, the purpose of $\heightRandom(\boldsymbol{x})$
is to add some stochastic topographic features to the top surfaces.

We choose to let $\heightRandom(\boldsymbol{x})$ model 
offshore sand ridges \cite{nil12:imp} with a given amplitude, width, and
length. These features can be modelled with a variogram with different ranges
in $x$- and $y$-direction, see the documentation file
\docVariogram{} for more information.

\section{Python script}
A Python script has been developed to generate a VESA
\cite{gas09:ver} input file for the model described in Section \ref{sec:introduction}.
Currently, only a single VESA input files is generated:
\begin{itemize}
\item \vesaInFile{REFHT} : The (negative) $z$-coordinate of each grid point of
  the generated top surface, i.e. $\heightTop$ as defined in Section
  \ref{sec:adding-stoch-feat}. 
\end{itemize}
\subsection{Input parameter file}
The input parameters to the Python script is given by a JSON input 
file. Here is a sample template file:
\begin{lstlisting}[language=json,firstnumber=1]
"domain" : {
   "curvature" : 0.005,
   "tilt_angle" : 0.477
},
"num_std_dev_confidence_delta_h" : 3,
"nx" : 300,
"ny" : 600,
"output_dir" : ".",
"output_thickness" : true,
"range_x" : 10000,
"range_y" : 80000,
"requested_delta_h" : 50,
"reservoir_depth" : 1000,
"reservoir_thickness" : 100,
"reservoir_xy_size" : [60000, 30000],
"seed" : 0,
"variogram" : {
   "type" : "exponential",
},
\end{lstlisting}
Here,
\begin{itemize}
\item \jsonParam{domain['curvature']} is the curvature of the domain according
  to \eqref{eq:2}. 
\item \jsonParam{domain['tilt\_angle']} is the domain tilt angle according
  to \eqref{eq:1}. 
\item \jsonParam{requested\_delta\_h} is the desired amplitude
  in the stochastic field $\heightRandom(\boldsymbol{x})$, see Section \ref{sec:adding-stoch-feat}.
The probability that a sampled value of $\heightRandom(\boldsymbol{x})$,  will
actually lie in the interval 
  $[ - \jsonParam{requested\_delta\_h}, \jsonParam{requested\_delta\_h} ]$
  is determined by the parameter 
  $N_{\stdev} := \text{\jsonParam{num\_std\_dev\_confidence\_delta\_h}}$. 
  The standard deviation \stdev{} of $\heightRandom(\boldsymbol{x})$ is set such that
  \begin{align}
    \label{eq:3}
    \stdev = \frac{\jsonParam{requested\_delta\_h}}{N_{\stdev}}
  \end{align}
  Hence, if for example
  $N_{\sigma}$ is set to 3, the probability that a sample from $\heightRandom$
  will be within the 
  interval $[ - \jsonParam{requested\_delta\_h}, \jsonParam{requested\_delta\_h} ]$
  is 0.9973 according to
  cumulative distribution function of the normal distribution, see
  \cite{joh10:sta}.  

  \item  The parameter \jsonParam{reservoir\_xy\_size} is an array of two
    elements where the first is the $x$-size (the length) of the reservoir, and the
    second specifies the $y$-size (the width) of the reservoir.

  \item The parameters \jsonParam{nx} and  \jsonParam{ny} specifies the number
    of grid cells in the $x$- and $y$-directions, respectively. The size of a
    grid cell $(\Delta x, \Delta y)$ is then
    \begin{align*}
      \Delta x = (\jsonParam{x\_max} - \jsonParam{x\_min})/ \jsonParam{nx}\\
      \Delta y = (\jsonParam{y\_max} - \jsonParam{y\_min})/ \jsonParam{ny}
    \end{align*}
    and the generated surface heights corresponds to the grid nodes, that is:
    to the cell centers (not the grid nodes/vertices).

\item \jsonParam{output\_dir} contains the directory name where the output
  files will be written. The default value (if the parameter is not given) is
  the current directory, ``\hltexttt{.}'' (i.e. a string consisting of a the single
  dot character). 

\item \jsonParam{output\_thickness} whether to output
the VESA input file \vesaInFile{H} containing the thickness of the reservoir at each
  grid point.
  The default value (if the parameter is not given) is not to output this
  file.

\item \jsonParam{range\_x} and \jsonParam{range\_y} are the range in the $x$-
  and $y$-direction of the variogram model, see the documentation file
  \docVariogram{} for more information.

\item \jsonParam{reservoir\_depth} specifies the depth (below sea level) of
  the lowest point of 
  the bottom surface. This number should be a positive number, although its
  $z$-coordinate will be its negative value.

\item \jsonParam{reservoir\_thickness} the thickness of the reservoir. This
  value is used to generate the output file \vesaInFile{H}.

\item \jsonParam{seed} The seed for the pseudo-random generator. If the value
  is not set (omitting the parameter from the file), or if 
  $\jsonParam{seed} = 0$, then the seed will not be set, resulting in a different
  realization for each run (according to the documentation for
  \hltexttt{numpy.random.RandomState}). Note: For debugging purposes, one can set
  \jsonParam{seed} to an integer not equal zero such that the same realization
  will be produced for consecutive runs.

\item \jsonParam{variogram['type']} the type of variogram to use when
  generating the covariance matrix, see the documentation file
  \docVariogram{} for more information.

\end{itemize}
\subsection{A sample case}
In this case we consider the following input parameters
\begin{lstlisting}[language=json,firstnumber=1]
"domain" : {
   "curvature" : 0.005,
   "tilt_angle" : 0.477
},
"num_std_dev_confidence_delta_h" : 3,
"nx" : 200,
"ny" : 125,
"range_x" : 5000,
"range_y" : 40000,
"requested_delta_h" : 50,
"reservoir_depth" : 1000,
"reservoir_thickness" : 100,
"variogram" : {
   "type" : "spherical",
},
"reservoir_xy_size" : [50000, 25000]
\end{lstlisting}
\subsubsection{Curvature}
We note that the domain width \domainWidth{} is set to 25000 meters according
parameter \jsonParam{reservoir\_xy\_size}. Now the domain curvature
\domainCurvature{} is set to 0.005 according to
\jsonParam{domain['curvature']}. We can now calculate the radius of the
cylinder according to Equation \eqref{eq:2}:
\begin{align}
  \label{eq:5}
  \domainCylRadius = \frac{\domainWidth}{2\domainCurvature} = 
   \frac{50000}{2\times 0.005} = 5,000,000
\end{align}
Refer to Figure \ref{fig:fcyl}, the line segment from $\domainXA$ to
$\domainXB$ and the radius line from the 
center of the cylinder to $\domainXB$ creates and angle $\theta$ that is given
by
\begin{align}
  \label{eq:6}
  \theta = \cos^{-1} \frac{\|\domainXB - \domainXA\|}{2r} = \cos^{-1}
  \domainCurvature = \cos^{-1} 0.005 \approx 89.71^{\circ}
\end{align}
The distance from the center of the cylinder to the line segment spanned by
$\domainXA$ and $\domainXB$ is now given by
\begin{align}
  \label{eq:7}
  h_0 = \domainCylRadius \sin \theta \approx 4999937.5,
\end{align}
Hence the maximum added height due to the curvature (that is: the maximum
height is attained at the midpoint
between $\domainXA$ and $\domainXB$) is
\begin{align}
  \label{eq:9}
  \Delta h_{\text{max}} = \domainCylRadius - h_0 = 5,000,000 - 4,999,937.5 =
  62.5 \;\text{m}.
\end{align}
\subsubsection{Tilt}
We will compute the domain height displacement \domainAngleHeight{} in
 Figure \ref{fig:dom}. From \jsonParam{reservoir\_xy\_size} we see that the
 domain length \domainLength{} is 50,000 meters. From
 \jsonParam{domain['tilt\_angle']} we see that the tilt angle \tiltAngle{} is
 0.477 degrees. We now have 
\begin{align}
  \label{eq:4}
  \domainAngleHeight = \domainLength \sin
  \left(
  \frac{\pi\tiltAngle}{180}
  \right) = 50000 \times \sin 
  \left(
  \frac{\pi \times 0.477}{180}
  \right) \approx 416.3 \; \text{m} 
\end{align}

%\begin{figure}
%  \centering
%  \begin{ovpc}{.8}{vesa/co2sat/5_40.png}
%  \end{ovpc}
%  \caption{a}
%  \label{fig:co2sat}
%\end{figure}
\bibliographystyle{plain}
\begin{thebibliography}{1}

\bibitem{gas09:ver}
S.E. Gasda, J.M. Nordbotten, and M.A. Celia.
\newblock Vertical equilibrium with sub-scale analytical methods for geological
  {CO2} sequestration.
\newblock {\em Comput. Geosci.}, 13(4):469--481, 2009.

\bibitem{joh10:sta}
R.A. Johnson and G.K. Bhattacharyya.
\newblock {\em Statistics. Principles and methods}.
\newblock John Wiley \& Sons, 6th edition, 2010.

\bibitem{nil12:imp}
H.M. Nilsen, A.R. Syversveen, K.-A. Lie, J.~Tveranger, and J.M. Nordbotten.
\newblock Impact of top-surface morphology on {CO2} storage capacity.
\newblock {\em International Journal of Greenhouse Gas Control}, 11:221--235,
  2012.

\end{thebibliography}

\end{document}
