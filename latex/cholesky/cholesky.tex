\documentclass[11pt,a4paper]{amsart}
\usepackage[numbers]{natbib}
\usepackage{stix}
\usepackage{siunitx}
\usepackage{booktabs}
\usepackage{overpic}
\usepackage{xspace}
\usepackage{bm}
\usepackage[implicit=false,colorlinks=true,urlcolor=blue]{hyperref}
\usepackage{soul}
\usepackage{xparse}
\usepackage{xcolor}
\definecolor{Light}{gray}{.90}
\sethlcolor{Light}
\usepackage{amsmath}

\newcommand{\cholMatrix}{\ensuremath{\boldsymbol{L}}}
\newcommand{\cholMatrixT}{\ensuremath{\boldsymbol{L}^{\text{T}}}}
\newcommand{\covMatrix}{\ensuremath{\boldsymbol{C}}}
\newcommand{\covMatrixElem}{\ensuremath{c}}
\newcommand{\cotwo}{\ensuremath{\text{CO}_2}}
\DeclareMathOperator{\covarianceFuncXY}{Cov}
\DeclareMathOperator{\covarianceFuncH}{C}
\newcommand{\deltaH}{\ensuremath{\Delta h}}
\newcommand{\displacement}{\ensuremath{\boldsymbol{h}}}
\newcommand{\distance}{\ensuremath{h}}
\newcommand{\distMatrix}{\ensuremath{\boldsymbol{H}}}
\newcommand{\docMain}{\hltexttt{main.pdf}}
\newcommand{\docSetup}{\hltexttt{setup.pdf}}
\newcommand{\docVariogram}{\hltexttt{variogram.pdf}}
\newcommand{\docCholesky}{\hltexttt{cholesky.pdf}}
\newcommand{\docsFolder}{\hltexttt{docs}}
\newcommand{\domainCylCenter}{\ensuremath{P}}
\newcommand{\domainCurvature}{\ensuremath{\gamma}}
\newcommand{\domainCylRadius}{\ensuremath{r}}
\newcommand{\domainWidth}{\ensuremath{w}}
\newcommand{\domainLength}{\ensuremath{l}}
\newcommand{\domainHeight}{\ensuremath{h}}
\newcommand{\domainAngleHeight}{\ensuremath{H}}
\newcommand{\domainXA}{\ensuremath{\boldsymbol{x}_1}}
\newcommand{\domainXB}{\ensuremath{\boldsymbol{x}_2}}
\newcommand{\domainPoint}{\ensuremath{\boldsymbol{x}}}
\DeclareMathOperator{\expectedValue}{E}
\newcommand{\heightBottom}{\ensuremath{\mathcal{H}_{\text{ref}}}}
\newcommand{\heightRandomReal}{\ensuremath{\hat{h}}}
\newcommand{\heightRandom}{\ensuremath{\tilde{H}}}
\newcommand{\heightTop}{\ensuremath{\mathcal{H}_{\text{top}}}}
\newcommand{\heightTopBase}{\ensuremath{\mathcal{H}_{\text{top,0}}}}
\newcommand{\hltexttt}[1]{\texttt{\hl{#1}}}
\newcommand{\hmin}{\ensuremath{h_{\text{min}}}}
\newcommand{\hmax}{\ensuremath{h_{\text{max}}}}
\newcommand{\identityMatrix}{\ensuremath{\boldsymbol{I}}}
\newcommand{\jsonInputParamFile}{\hltexttt{input\_params.json}}
\newcommand{\jsonParam}[1]{\ensuremath{\text{\hltexttt{#1}}}}
\newcommand{\mainScript}{\hltexttt{gen\_realizations\_and\_run\_vesa.py}}
\newcommand{\mainScriptKJ}{\hltexttt{gen\_vesa\_input\_file.py}}
\newcommand{\mean}{\ensuremath{\mu}}
\newcommand{\numDomainPoints}{\ensuremath{N}}
\newcommand{\paramAmp}{\scriptParam{requested\_delta\_h}}
\newcommand{\paramNumReal}{\scriptParam{num\_real}}
\newcommand{\paramStd}{\scriptParam{std}}
\newcommand{\pythonModule}[1]{\hltexttt{#1}}
\newcommand{\pythonPip}{\hltexttt{pip}}
\newcommand{\pythScriptName}{\hltexttt{gen\_vesa\_input\_files.py}}
\newcommand{\radDist}{\ensuremath{r}}
\newcommand{\randVarZ}{\ensuremath{\tilde{Z}}}
\newcommand{\randVecZ}{\ensuremath{\tilde{\boldsymbol{Z}}}}
\newcommand{\randVecz}{\ensuremath{\tilde{\boldsymbol{z}}}}
\newcommand{\range}{\ensuremath{\kappa}}
\newcommand{\rangeX}{\scriptParam{range\_x}}
\newcommand{\rangeY}{\scriptParam{range\_y}}
\newcommand{\reservoirThickness}{\ensuremath{\mathcal{W}}}
\newcommand{\scriptParam}[1]{\ensuremath{\hltexttt{#1}}}
\newcommand{\shellCommand}[1]{\ensuremath{\text{\hltexttt{#1}}}}
\newcommand{\stdev}{\ensuremath{\sigma}}
\newcommand{\tiltAngle}{\ensuremath{\alpha}}
\newcommand{\unit}[1]{\ensuremath{[\si{#1}]}}
\newcommand{\variance}{\ensuremath{\sigma^2}}
\newcommand{\varianceFunc}{\ensuremath{\text{Var}}}
\newcommand{\variogram}{\ensuremath{\gamma}}
\newcommand{\variogramVariance}{\ensuremath{\sigma_v^2}}
\newcommand{\vesaVariable}[1]{\ensuremath{\text{\hltexttt{#1}}}}
\newcommand{\vesaGitHub}{\url{https://github.com/NORCE-Energy/vesa}}
\newcommand{\vesaOutFile}[1]{\hltexttt{#1\_out}}

\DeclareDocumentCommand \vesaInFile { m o } {
  \IfNoValueTF {#2}%
    {\hltexttt{#1.IN}}%
    {\hltexttt{#1.IN}[#2]}%
}


\newcommand{\mMatrix}{\ensuremath{\boldsymbol{M}}}
\newcommand{\numSamples}{\ensuremath{n}}
\newcommand{\xVar}{\ensuremath{\tilde{x}}}
\newcommand{\xVec}{\ensuremath{\tilde{\boldsymbol{x}}}}
\newenvironment{ovpc}[2]{\begin{overpic}[width=#1\textwidth]{../fig/#2}}{\end{overpic}}
\begin{document}
\title{Cholesky factorization for generating correlated variables}
\date{\today}
\maketitle
\section{Introduction}
The Cholesky decomposition is commonly used in geostatistical reservoir modelling for
generating realizations of spatially correlated random variables.

For a real positive definite matrix $\mMatrix$, the Cholesky decomposition is
the product of a lower triangular matrix $\cholMatrix$ and its
transpose, i.e.: $\mMatrix = \cholMatrix\cholMatrix^T$, \cite{gol96:mat}.

In our application to geostatistics, we assume we are given a
$\numSamples\times \numSamples$ covariance matrix
$\covMatrix$ generated from a variogram model of a secondary stationary random
field and an isotropic covariance function.
Here, $\numSamples$ is the number of
sampling points points. A valid variogram model will give a
positive definite $\covMatrix$ \cite{chi12:geo},
hence it will always be possible to determine a lower triangular
$\numSamples\times \numSamples$ matrix $\cholMatrix{}$ such that
\begin{align}
  \label{eq:6}
  \covMatrix = \cholMatrix \cholMatrixT. 
\end{align}

A widely used method for sampling a $\numSamples \times 1$ vector
from the multivariate normal distribution with mean $\mean$
  and covariance matrix $\covMatrix$ works as follows \cite{gen09:com}:

Consider a random $\numSamples\times 1$ vector $\xVec$ 
consisting of uncorrelated random variables
from the standard normal distribution, such that each random variable
$\xVar_i$, $i = 1, \ldots, \numSamples$, of $\xVec$
has a zero mean and a unit variance.

It follows that the expected value of any pairs of these variables is
\begin{align}
  \label{eq:3}
  \expectedValue(\xVar_i\xVar_j)=\delta_{ij},
\end{align}
where $\delta_{ij}$ is the Kronecker delta,
and hence the covariance matrix of $\xVec$ is the identity matrix
\begin{align}
  \label{eq:4}
  \expectedValue(\xVec\xVec^T) = \identityMatrix
\end{align}

Now define a new random vector $\randVecz = \mean+\cholMatrix \xVec$ and observe
that the covariance 
matrix of $\randVecz - \mean$ is given by
\begin{multline}
  \label{eq:5}
  %\expectedValue(\randVecz\randVecz^T) =
  \expectedValue((\cholMatrix \xVec)(\cholMatrix \xVec)^T) =
  \expectedValue(\cholMatrix \xVec\xVec^T\cholMatrix^T) =
  \cholMatrix\expectedValue(\xVec\xVec^T)\cholMatrix^T = 
  \cholMatrix \identityMatrix \cholMatrix^T = 
  \cholMatrix \cholMatrix^T = \covMatrix 
\end{multline}
So we have now a random vector $\randVecz$ that is normal distributed with
mean $\mean$ and covariance as given by $\covMatrix$. To sample from this vector,
we simply draw a sample from $\xVec$ and multiply the sample vector on the
left by $\cholMatrix$.


\bibliographystyle{plain}
\begin{thebibliography}{1}

\bibitem{chi12:geo}
J.-P. Chil{\`{e}}s and P.~Delfiner.
\newblock {\em {Geostatistics : Modeling Spatial Uncertainty}}.
\newblock Wiley, 2012.

\bibitem{gen09:com}
J.E. Gentle.
\newblock {\em Computational statistics}.
\newblock Springer, 2009.

\bibitem{gol96:mat}
G.H. Golub and C.F. Van~Loan.
\newblock {\em Matrix computations}.
\newblock Johns Hopkins studies in the mathematical sciences vol. 3. Johns
  Hopkins University Press, Baltimore, third edition, 1996.

\end{thebibliography}

\end{document}
