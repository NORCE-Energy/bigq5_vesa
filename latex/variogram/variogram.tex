\documentclass[11pt,a4paper]{amsart}
\usepackage[numbers]{natbib}
\usepackage{stix}
\usepackage{siunitx}
\usepackage{booktabs}
\usepackage{overpic}
\usepackage{xspace}
\usepackage{bm}
\usepackage[implicit=false,colorlinks=true,urlcolor=blue]{hyperref}
\usepackage{soul}
\usepackage{xparse}
\usepackage{xcolor}
\definecolor{Light}{gray}{.90}
\sethlcolor{Light}
\usepackage{amsmath}

\newcommand{\cholMatrix}{\ensuremath{\boldsymbol{L}}}
\newcommand{\cholMatrixT}{\ensuremath{\boldsymbol{L}^{\text{T}}}}
\newcommand{\covMatrix}{\ensuremath{\boldsymbol{C}}}
\newcommand{\covMatrixElem}{\ensuremath{c}}
\newcommand{\cotwo}{\ensuremath{\text{CO}_2}}
\DeclareMathOperator{\covarianceFuncXY}{Cov}
\DeclareMathOperator{\covarianceFuncH}{C}
\newcommand{\deltaH}{\ensuremath{\Delta h}}
\newcommand{\displacement}{\ensuremath{\boldsymbol{h}}}
\newcommand{\distance}{\ensuremath{h}}
\newcommand{\distMatrix}{\ensuremath{\boldsymbol{H}}}
\newcommand{\docMain}{\hltexttt{main.pdf}}
\newcommand{\docSetup}{\hltexttt{setup.pdf}}
\newcommand{\docVariogram}{\hltexttt{variogram.pdf}}
\newcommand{\docCholesky}{\hltexttt{cholesky.pdf}}
\newcommand{\docsFolder}{\hltexttt{docs}}
\newcommand{\domainCylCenter}{\ensuremath{P}}
\newcommand{\domainCurvature}{\ensuremath{\gamma}}
\newcommand{\domainCylRadius}{\ensuremath{r}}
\newcommand{\domainWidth}{\ensuremath{w}}
\newcommand{\domainLength}{\ensuremath{l}}
\newcommand{\domainHeight}{\ensuremath{h}}
\newcommand{\domainAngleHeight}{\ensuremath{H}}
\newcommand{\domainXA}{\ensuremath{\boldsymbol{x}_1}}
\newcommand{\domainXB}{\ensuremath{\boldsymbol{x}_2}}
\newcommand{\domainPoint}{\ensuremath{\boldsymbol{x}}}
\DeclareMathOperator{\expectedValue}{E}
\newcommand{\heightBottom}{\ensuremath{\mathcal{H}_{\text{ref}}}}
\newcommand{\heightRandomReal}{\ensuremath{\hat{h}}}
\newcommand{\heightRandom}{\ensuremath{\tilde{H}}}
\newcommand{\heightTop}{\ensuremath{\mathcal{H}_{\text{top}}}}
\newcommand{\heightTopBase}{\ensuremath{\mathcal{H}_{\text{top,0}}}}
\newcommand{\hltexttt}[1]{\texttt{\hl{#1}}}
\newcommand{\hmin}{\ensuremath{h_{\text{min}}}}
\newcommand{\hmax}{\ensuremath{h_{\text{max}}}}
\newcommand{\identityMatrix}{\ensuremath{\boldsymbol{I}}}
\newcommand{\jsonInputParamFile}{\hltexttt{input\_params.json}}
\newcommand{\jsonParam}[1]{\ensuremath{\text{\hltexttt{#1}}}}
\newcommand{\mainScript}{\hltexttt{gen\_realizations\_and\_run\_vesa.py}}
\newcommand{\mainScriptKJ}{\hltexttt{gen\_vesa\_input\_file.py}}
\newcommand{\mean}{\ensuremath{\mu}}
\newcommand{\numDomainPoints}{\ensuremath{N}}
\newcommand{\paramAmp}{\scriptParam{requested\_delta\_h}}
\newcommand{\paramNumReal}{\scriptParam{num\_real}}
\newcommand{\paramStd}{\scriptParam{std}}
\newcommand{\pythonModule}[1]{\hltexttt{#1}}
\newcommand{\pythonPip}{\hltexttt{pip}}
\newcommand{\pythScriptName}{\hltexttt{gen\_vesa\_input\_files.py}}
\newcommand{\radDist}{\ensuremath{r}}
\newcommand{\randVarZ}{\ensuremath{\tilde{Z}}}
\newcommand{\randVecZ}{\ensuremath{\tilde{\boldsymbol{Z}}}}
\newcommand{\randVecz}{\ensuremath{\tilde{\boldsymbol{z}}}}
\newcommand{\range}{\ensuremath{\kappa}}
\newcommand{\rangeX}{\scriptParam{range\_x}}
\newcommand{\rangeY}{\scriptParam{range\_y}}
\newcommand{\reservoirThickness}{\ensuremath{\mathcal{W}}}
\newcommand{\scriptParam}[1]{\ensuremath{\hltexttt{#1}}}
\newcommand{\shellCommand}[1]{\ensuremath{\text{\hltexttt{#1}}}}
\newcommand{\stdev}{\ensuremath{\sigma}}
\newcommand{\tiltAngle}{\ensuremath{\alpha}}
\newcommand{\unit}[1]{\ensuremath{[\si{#1}]}}
\newcommand{\variance}{\ensuremath{\sigma^2}}
\newcommand{\varianceFunc}{\ensuremath{\text{Var}}}
\newcommand{\variogram}{\ensuremath{\gamma}}
\newcommand{\variogramVariance}{\ensuremath{\sigma_v^2}}
\newcommand{\vesaVariable}[1]{\ensuremath{\text{\hltexttt{#1}}}}
\newcommand{\vesaGitHub}{\url{https://github.com/NORCE-Energy/vesa}}
\newcommand{\vesaOutFile}[1]{\hltexttt{#1\_out}}

\DeclareDocumentCommand \vesaInFile { m o } {
  \IfNoValueTF {#2}%
    {\hltexttt{#1.IN}}%
    {\hltexttt{#1.IN}[#2]}%
}


\newenvironment{ovpc}[2]{\begin{overpic}[width=#1\textwidth]{../fig/#2}}{\end{overpic}}
\begin{document}
\title{Variogram}
\date{\today}
\maketitle
\section{Introduction}
\label{sec:introduction}
Assume we want to model a reservoir top surface. As a starting point for the
model, we assume that the depth (i.e. the negative $z$-coordinate assuming
$z=0$ corresponds to sea level) of the top surface of the reservoir is
given as
\begin{align}
  \label{eq:17}
  \heightTop(\boldsymbol{x}) = \heightTopBase(\boldsymbol{x})
  + \heightRandomReal(\boldsymbol{x})
\end{align}
here $\boldsymbol{x} = (x, y)$ is a point in the $xy$-plane, and
where $\heightRandomReal(\boldsymbol{x})$ is a realization of a random field
$\heightRandom(\boldsymbol{x})$ with a mean of zero at any point and a
standard deviation that is much smaller than the reservoir thickness
$\reservoirThickness(\boldsymbol{x})$.
Also, note the bottom surface of the reservoir is now given by
\begin{align}
  \label{eq:20}  
  \heightBottom(\boldsymbol{x}) = \heightTop(\boldsymbol{x})
  - \reservoirThickness(\boldsymbol{x})
\end{align}
Hence, $\heightRandom(\boldsymbol{x})$
is used to add some randomness to the top surfaces.
We now consider how to model the random field $\heightRandom(\boldsymbol{x})$.
The domain is discretized into points $\domainPoint_i$, $i=1, \ldots,
\numDomainPoints$, and 
we consider the random variables $\heightRandom_i = \heightRandom(\domainPoint_i)$.

The collection of random variables $\{ \heightRandom_i
\}_{i=1}^{\numDomainPoints}$ is called a 
random vector. 
A random vector is called stationary if the probability  
distribution of each component is the same.

\section{Variogram and covariance}
The definition of the theoretical variogram, \variogram{}, is based on
regionalized random variables $\randVarZ(\boldsymbol{x})$ and
$\randVarZ(\boldsymbol{x}+\boldsymbol{h})$, 
where $\boldsymbol{x}$ and $\boldsymbol{x}+\boldsymbol{h}$ represent the
spatial positions separated by a vector $\boldsymbol{h}$ \cite{bac11:var}
 \begin{align}
   \label{eq:1}
   \variogram(\boldsymbol{h}) = \frac{1}{2}\expectedValue
   \left[
   \left[
   \randVarZ(\boldsymbol{x}+\boldsymbol{h}) - \randVarZ(\boldsymbol{x})
   \right]^2
   \right] = \frac{1}{2} \varianceFunc
   \left[
   \randVarZ(\boldsymbol{x}+\boldsymbol{h}) - \randVarZ(\boldsymbol{x})
   \right]
 \end{align}
Here, $\randVarZ(\boldsymbol{x})$ and $\randVarZ(\boldsymbol{x} + \boldsymbol{h})$
 denote random variables, and $\expectedValue[\cdot]$ denotes the expected
 value, and
the variance $\varianceFunc(\randVarZ(\boldsymbol{x}))$ is defined as
 \begin{align}
   \label{eq:10}
   \varianceFunc(\randVarZ(\boldsymbol{x}))= \expectedValue
   \left[
   \left(
   \randVarZ(\boldsymbol{x}) - \expectedValue
   \left[
   \randVarZ(\boldsymbol{x})
   \right]
   \right)^2
   \right]
 \end{align}

 Further, the covariance between the random variables $\randVarZ(\boldsymbol{x})$ and
$\randVarZ(\boldsymbol{x} + \boldsymbol{h})$ is defined as the expected value
of the product of their deviations from their individual expected values
\begin{multline}
  \label{eq:9}
  \covarianceFuncXY (\randVarZ(\boldsymbol{x}),\randVarZ(\boldsymbol{x} +
  \boldsymbol{h}))=\\
  \expectedValue
  \left[
    \left\{
  \randVarZ(\boldsymbol{x}) - \expectedValue
  \left[
  \randVarZ(\boldsymbol{x})
  \right]
\right\} \times
    \left\{
   \randVarZ(\boldsymbol{x} + \boldsymbol{h}) - \expectedValue
  \left[
  \randVarZ(\boldsymbol{x} + \boldsymbol{h})
  \right]
\right\}
  \right]
\end{multline}

If the first two moments of the variable $\randVarZ$ are invariant under
translation, it is called second order stationary \cite{chi12:geo}. In this
case, the mean is constant and the covariance function only depends on the
separation $\boldsymbol{h}$.

It follows from the variance addition theorem \cite{pap91:pro} that
\begin{multline}
  \label{eq:3}
  \varianceFunc
   \left[
   \randVarZ(\boldsymbol{x}+\boldsymbol{h}) - \randVarZ(\boldsymbol{x})
   \right]
  =\\
   \varianceFunc
   \left[
   \randVarZ(\boldsymbol{x}+\boldsymbol{h})
  \right] +
     \varianceFunc
   \left[
   \randVarZ(\boldsymbol{x})
  \right] - 2 \covarianceFuncXY 
  \left[
  \randVarZ(\boldsymbol{x}+\boldsymbol{h}), \randVarZ(\boldsymbol{x})
  \right],
\end{multline}
and since the variance is independent of the position, this is equal to
\begin{align}
  \label{eq:4}
  2 \varianceFunc
   \left[
   \randVarZ(\boldsymbol{x})
  \right] - 2\covarianceFuncXY 
  \left[
  \randVarZ(\boldsymbol{x}+\boldsymbol{h}), \randVarZ(\boldsymbol{x})
  \right],
\end{align}

Let us define
\begin{align}
  \label{eq:5}
  \variance = \varianceFunc
  \left[
  \randVarZ(\boldsymbol{x})
  \right]
\end{align}
and and since covariance is independent of position we also define
\begin{align}
  \label{eq:6}
  \covarianceFuncH(\boldsymbol{h}) = \covarianceFuncXY
  \left[
  \randVarZ(\boldsymbol{x}+\boldsymbol{h}), \randVarZ(\boldsymbol{x})
  \right],
\end{align}
Using \eqref{eq:1}, \eqref{eq:3}, and \eqref{eq:4}, we thus have
\begin{align}
  \label{eq:7}
  \variogram(\boldsymbol{h}) = \variance -  \covarianceFuncH(\boldsymbol{h})
\end{align}
Here, $\variance$ is identical with the sill of the variogram.
The sill is the limit of $\variogram(\boldsymbol{h})$
for $\|\boldsymbol{h}\| \rightarrow \infty$
so that the covariance function tends to zero.

If furthermore the covariance function $\covarianceFuncH(\boldsymbol{h})$
only depends on the length $\|\boldsymbol{h}\|$ 
of the vector $\boldsymbol{h}$ and not on its orientation, we have
\begin{align}
  \label{eq:15}
  \variogram(h) &= \variance -  \covarianceFuncH(h), \qquad \text{or}\\
  \label{eq:8}
  \covarianceFuncH(h) &= \variance -  \variogram(h)
\end{align}
\section{Covariance model}
We compare some covariance models for $h\geq 0$.
The value of the distance $h$ where the model first flattens out (and become
zero) is known as the range $\range$.

\subsection{Model 1: $\sin x/x$}
The model is given by
\begin{align}
  \label{eq:2}
  \covarianceFuncH_1(h) =
  \begin{cases}
    \variance & h = 0,\\
    \variance\frac{\sin (\pi h )/ \range}{(\pi h)/\range}, & 0 < h < \range,\\
    0, & h \geq \range
  \end{cases}
\end{align}
%Note: that the covariance function becomes zero at its range,
%i.e. $\covarianceFuncH(\range) = 0$.
Note that this function is probably not (how to prove this?) a covariance function
as defined in \cite{chi12:geo}. Numerical test has also shown that in some cases
the resulting covariance matrix is not positive definite.

\subsection{Model 2:  spherical}
The model is given by \cite{chi12:geo}
\begin{align}
  \label{eq:11}
  \covarianceFuncH_2(h) =
  \begin{cases}
    \variance
    \left(
      1 - \left(
        \frac{3h}{2\range} - \frac{h^3}{2\range^3}
      \right)
    \right), &h<\range\\
    0, & h\geq\range
  \end{cases}
\end{align}
\subsection{Model 3: triangular}
The model is given by 
\begin{align}
  \label{eq:18}
  \covarianceFuncH_3(h) =
  \begin{cases}
    \variance \left(1 - \frac{h}{\range}\right), &h<\range\\
    0, & h\geq\range
  \end{cases}
\end{align}
\subsection{Model 4: exponential}
The model is given by 
\begin{align}
  \label{eq:19}  
  \covarianceFuncH_4(h) =
    \variance \exp(-\frac{8h}{\range}), \qquad h\geq 0
\end{align}
Note that the variogram is not zero at $h \geq \range$. We have
\begin{align}
  \label{eq:16}
  \covarianceFuncH_4(\range) = \variance e^{-8}\approx \variance (3.354 \times 10^{-4})
\end{align}
\subsection{Comparison of the models}
The models are compared in Figure \ref{fig:cov}.
\begin{figure}
  \centering
  \begin{ovpc}{.8}{variogram/covariance_cmp.png}
  \end{ovpc}
  \caption{Comparison of the four models, using a range of $\range = 20$ and a
    variance of $\variance = 46.667^2 \approx 2177.8$ }
  \label{fig:cov}
\end{figure}
\section{Generation of the covariance matrix}
First we define the distances
\begin{align}
  \label{eq:13}
  \distance_{ij} = \| \domainPoint_i - \domainPoint_j \|, \qquad
  i=1,\ldots, \numDomainPoints, \quad j=1,\ldots,\numDomainPoints,
\end{align}
where $\domainPoint_i$ is a node (discrete point) in the solution domain as
described in Section \ref{sec:introduction}.
Now define the $\numDomainPoints \times \numDomainPoints$ distance matrix
\begin{align}
  \label{eq:14}
  \distMatrix = \{ \distance_{ij} \}_{i=1,\ldots,\numDomainPoints, \; j=1,
  \ldots, \numDomainPoints}.
\end{align}
Then the covariance matrix is given by
\begin{align}
  \label{eq:12}
  \covMatrix = \{ \covarianceFuncH(\distance_{ij})
  \}_{i=1,\ldots,\numDomainPoints, \; j=1,  
  \ldots, \numDomainPoints}.
,
\end{align}
where $\covarianceFuncH(\cdot)$ is the covariance function as defined in
Equation \eqref{eq:8}.
\section{Coordinate transformations}
A different covariance range $\range$ in the $x$ and $y$ direction can be
incorporated in the covariance matrix in \eqref{eq:12} by scaling e.g. the $y$
coordinate before computing the distances in \eqref{eq:13}.

\bibliographystyle{plain}
\begin{thebibliography}{1}

\bibitem{bac11:var}
M.~Bachmaier and M.~Backes.
\newblock Variogram or semivariogram? variance or semivariance? allan variance
  or introducing a new term?
\newblock {\em Math. Geosci.}, 43:735--740, 2011.

\bibitem{chi12:geo}
J.-P. Chil{\`{e}}s and P.~Delfiner.
\newblock {\em {Geostatistics : Modeling Spatial Uncertainty}}.
\newblock Wiley, 2012.

\bibitem{pap91:pro}
A.~Papoulis.
\newblock {\em Probability, random variables, and stochastic processes}.
\newblock McGraw-Hill, third edition, 1991.

\end{thebibliography}

\end{document}
